% Options for packages loaded elsewhere
\PassOptionsToPackage{unicode}{hyperref}
\PassOptionsToPackage{hyphens}{url}
%
\documentclass[
]{article}
\usepackage{amsmath,amssymb}
\usepackage{lmodern}
\usepackage{iftex}
\ifPDFTeX
  \usepackage[T1]{fontenc}
  \usepackage[utf8]{inputenc}
  \usepackage{textcomp} % provide euro and other symbols
\else % if luatex or xetex
  \usepackage{unicode-math}
  \defaultfontfeatures{Scale=MatchLowercase}
  \defaultfontfeatures[\rmfamily]{Ligatures=TeX,Scale=1}
\fi
% Use upquote if available, for straight quotes in verbatim environments
\IfFileExists{upquote.sty}{\usepackage{upquote}}{}
\IfFileExists{microtype.sty}{% use microtype if available
  \usepackage[]{microtype}
  \UseMicrotypeSet[protrusion]{basicmath} % disable protrusion for tt fonts
}{}
\makeatletter
\@ifundefined{KOMAClassName}{% if non-KOMA class
  \IfFileExists{parskip.sty}{%
    \usepackage{parskip}
  }{% else
    \setlength{\parindent}{0pt}
    \setlength{\parskip}{6pt plus 2pt minus 1pt}}
}{% if KOMA class
  \KOMAoptions{parskip=half}}
\makeatother
\usepackage{xcolor}
\usepackage[margin=1in]{geometry}
\usepackage{color}
\usepackage{fancyvrb}
\newcommand{\VerbBar}{|}
\newcommand{\VERB}{\Verb[commandchars=\\\{\}]}
\DefineVerbatimEnvironment{Highlighting}{Verbatim}{commandchars=\\\{\}}
% Add ',fontsize=\small' for more characters per line
\usepackage{framed}
\definecolor{shadecolor}{RGB}{248,248,248}
\newenvironment{Shaded}{\begin{snugshade}}{\end{snugshade}}
\newcommand{\AlertTok}[1]{\textcolor[rgb]{0.94,0.16,0.16}{#1}}
\newcommand{\AnnotationTok}[1]{\textcolor[rgb]{0.56,0.35,0.01}{\textbf{\textit{#1}}}}
\newcommand{\AttributeTok}[1]{\textcolor[rgb]{0.77,0.63,0.00}{#1}}
\newcommand{\BaseNTok}[1]{\textcolor[rgb]{0.00,0.00,0.81}{#1}}
\newcommand{\BuiltInTok}[1]{#1}
\newcommand{\CharTok}[1]{\textcolor[rgb]{0.31,0.60,0.02}{#1}}
\newcommand{\CommentTok}[1]{\textcolor[rgb]{0.56,0.35,0.01}{\textit{#1}}}
\newcommand{\CommentVarTok}[1]{\textcolor[rgb]{0.56,0.35,0.01}{\textbf{\textit{#1}}}}
\newcommand{\ConstantTok}[1]{\textcolor[rgb]{0.00,0.00,0.00}{#1}}
\newcommand{\ControlFlowTok}[1]{\textcolor[rgb]{0.13,0.29,0.53}{\textbf{#1}}}
\newcommand{\DataTypeTok}[1]{\textcolor[rgb]{0.13,0.29,0.53}{#1}}
\newcommand{\DecValTok}[1]{\textcolor[rgb]{0.00,0.00,0.81}{#1}}
\newcommand{\DocumentationTok}[1]{\textcolor[rgb]{0.56,0.35,0.01}{\textbf{\textit{#1}}}}
\newcommand{\ErrorTok}[1]{\textcolor[rgb]{0.64,0.00,0.00}{\textbf{#1}}}
\newcommand{\ExtensionTok}[1]{#1}
\newcommand{\FloatTok}[1]{\textcolor[rgb]{0.00,0.00,0.81}{#1}}
\newcommand{\FunctionTok}[1]{\textcolor[rgb]{0.00,0.00,0.00}{#1}}
\newcommand{\ImportTok}[1]{#1}
\newcommand{\InformationTok}[1]{\textcolor[rgb]{0.56,0.35,0.01}{\textbf{\textit{#1}}}}
\newcommand{\KeywordTok}[1]{\textcolor[rgb]{0.13,0.29,0.53}{\textbf{#1}}}
\newcommand{\NormalTok}[1]{#1}
\newcommand{\OperatorTok}[1]{\textcolor[rgb]{0.81,0.36,0.00}{\textbf{#1}}}
\newcommand{\OtherTok}[1]{\textcolor[rgb]{0.56,0.35,0.01}{#1}}
\newcommand{\PreprocessorTok}[1]{\textcolor[rgb]{0.56,0.35,0.01}{\textit{#1}}}
\newcommand{\RegionMarkerTok}[1]{#1}
\newcommand{\SpecialCharTok}[1]{\textcolor[rgb]{0.00,0.00,0.00}{#1}}
\newcommand{\SpecialStringTok}[1]{\textcolor[rgb]{0.31,0.60,0.02}{#1}}
\newcommand{\StringTok}[1]{\textcolor[rgb]{0.31,0.60,0.02}{#1}}
\newcommand{\VariableTok}[1]{\textcolor[rgb]{0.00,0.00,0.00}{#1}}
\newcommand{\VerbatimStringTok}[1]{\textcolor[rgb]{0.31,0.60,0.02}{#1}}
\newcommand{\WarningTok}[1]{\textcolor[rgb]{0.56,0.35,0.01}{\textbf{\textit{#1}}}}
\usepackage{longtable,booktabs,array}
\usepackage{calc} % for calculating minipage widths
% Correct order of tables after \paragraph or \subparagraph
\usepackage{etoolbox}
\makeatletter
\patchcmd\longtable{\par}{\if@noskipsec\mbox{}\fi\par}{}{}
\makeatother
% Allow footnotes in longtable head/foot
\IfFileExists{footnotehyper.sty}{\usepackage{footnotehyper}}{\usepackage{footnote}}
\makesavenoteenv{longtable}
\usepackage{graphicx}
\makeatletter
\def\maxwidth{\ifdim\Gin@nat@width>\linewidth\linewidth\else\Gin@nat@width\fi}
\def\maxheight{\ifdim\Gin@nat@height>\textheight\textheight\else\Gin@nat@height\fi}
\makeatother
% Scale images if necessary, so that they will not overflow the page
% margins by default, and it is still possible to overwrite the defaults
% using explicit options in \includegraphics[width, height, ...]{}
\setkeys{Gin}{width=\maxwidth,height=\maxheight,keepaspectratio}
% Set default figure placement to htbp
\makeatletter
\def\fps@figure{htbp}
\makeatother
\setlength{\emergencystretch}{3em} % prevent overfull lines
\providecommand{\tightlist}{%
  \setlength{\itemsep}{0pt}\setlength{\parskip}{0pt}}
\setcounter{secnumdepth}{-\maxdimen} % remove section numbering
\ifLuaTeX
  \usepackage{selnolig}  % disable illegal ligatures
\fi
\IfFileExists{bookmark.sty}{\usepackage{bookmark}}{\usepackage{hyperref}}
\IfFileExists{xurl.sty}{\usepackage{xurl}}{} % add URL line breaks if available
\urlstyle{same} % disable monospaced font for URLs
\hypersetup{
  pdftitle={DATA 621 HW 1},
  hidelinks,
  pdfcreator={LaTeX via pandoc}}

\title{DATA 621 HW 1}
\author{}
\date{\vspace{-2.5em}}

\begin{document}
\maketitle

\hypertarget{business-analytics-and-data-mining}{%
\section{\texorpdfstring{\textbf{Business Analytics and Data
Mining}}{Business Analytics and Data Mining}}\label{business-analytics-and-data-mining}}

\hypertarget{homework-1-assignment-requirements}{%
\subsection{Homework \#1 Assignment
Requirements}\label{homework-1-assignment-requirements}}

\hypertarget{overview}{%
\subsubsection{\texorpdfstring{\textbf{Overview}}{Overview}}\label{overview}}

In this homework assignment, you will explore, analyze and model a data
set containing approximately 2200 records. Each record represents a
professional baseball team from the years 1871 to 2006 inclusive. Each
record has the performance of the team for the given year, with all of
the statistics adjusted to match the performance of a 162 game season.

Your objective is to build a multiple linear regression model on the
training data to predict the number of wins for the team. You can only
use the variables given to you (or variables that you derive from the
variables provided).

Below is a short description of the variables of interest in the data
set:

\begin{longtable}[]{@{}
  >{\raggedright\arraybackslash}p{(\columnwidth - 4\tabcolsep) * \real{0.2727}}
  >{\centering\arraybackslash}p{(\columnwidth - 4\tabcolsep) * \real{0.4432}}
  >{\raggedleft\arraybackslash}p{(\columnwidth - 4\tabcolsep) * \real{0.2841}}@{}}
\toprule()
\begin{minipage}[b]{\linewidth}\raggedright
Variable Names
\end{minipage} & \begin{minipage}[b]{\linewidth}\centering
Definition
\end{minipage} & \begin{minipage}[b]{\linewidth}\raggedleft
Theoretical Effect
\end{minipage} \\
\midrule()
\endhead
INDEX & Identification Variable (do not use) & None \\
TARGET\_WINS & Number of wins & \$12 \\
TEAM\_BATTING\_H & Base Hits by batters (1B,2B,3B,HR) & Positive Impact
on Wins \\
TEAM\_BATTING\_2B & Doubles by batters (2B) & Positive Impact on Wins \\
TEAM\_BATTING\_3B & Triples by batters (3B) & Positive Impact on Wins \\
TEAM\_BATTING\_HR & Homeruns by batters (4B) & Positive Impact on
Wins \\
TEAM\_BATTING\_BB & Walks by batters & Positive Impact on Wins \\
TEAM\_BATTING\_HBP & Batters hit by pitch (get a free base) & Positive
Impact on Wins \\
TEAM\_BATTING\_SO & Strikeouts by batters & Negative Impact on Wins \\
TEAM\_BASERUN\_SB & Stolen bases & Positive Impact on Wins \\
TEAM\_BASERUN\_CS & Caught stealing & Negative Impact on Wins \\
TEAM\_FIELDING\_E & Errors & Negative Impact on Wins \\
TEAM\_FIELDING\_DP & Double Plays & Positive Impact on Wins \\
TEAM\_PITCHING\_BB & Walks allowed & Negative Impact on Wins \\
TEAM\_PITCHING\_H & Hits allowed & Negative Impact on Wins \\
TEAM\_PITCHING\_HR & Homeruns allowed & Negative Impact on Wins \\
TEAM\_PITCHING\_SO & Strikeouts by pitchers & Positive Impact on Wins \\
\bottomrule()
\end{longtable}

\textbf{Deliverable:}

\begin{itemize}
\tightlist
\item
  A write-up submitted in PDF format. Your write-up should have four
  sections. Each one is described below. You may assume you are
  addressing me as a fellow data scientist, so do not need to shy away
  from technical details.
\item
  Assigned predictions (the number of wins for the team) for the
  evaluation data set.
\item
  Include your R statistical programming code in an Appendix.
\end{itemize}

\textbf{Write Up:}

\begin{enumerate}
\def\labelenumi{\arabic{enumi}.}
\tightlist
\item
  \textbf{DATA EXPLORATION (25 Points)} Describe the size and the
  variables in the moneyball training data set. Consider that too much
  detail will cause a manager to lose interest while too little detail
  will make the manager consider that you aren't doing your job. Some
  suggestions are given below. Please do NOT treat this as a check list
  of things to do to complete the assignment. You should have your own
  thoughts on what to tell the boss. These are just ideas.
\end{enumerate}

\begin{enumerate}
\def\labelenumi{\alph{enumi}.}
\tightlist
\item
  Mean / Standard Deviation / Median
\item
  Bar Chart or Box Plot of the data
\item
  Is the data correlated to the target variable (or to other variables?)
\item
  Are any of the variables missing and need to be imputed ``fixed''?
\end{enumerate}

\begin{enumerate}
\def\labelenumi{\arabic{enumi}.}
\setcounter{enumi}{1}
\tightlist
\item
  \textbf{DATA PREPARATION (25 Points)} Describe how you have
  transformed the data by changing the original variables or creating
  new variables. If you did transform the data or create new variables,
  discuss why you did this. Here are some possible transformations.
\end{enumerate}

\begin{enumerate}
\def\labelenumi{\alph{enumi}.}
\tightlist
\item
  Fix missing values (maybe with a Mean or Median value)
\item
  Create flags to suggest if a variable was missing
\item
  Transform data by putting it into buckets
\item
  Mathematical transforms such as log or square root (or use Box-Cox)
\item
  Combine variables (such as ratios or adding or multiplying) to create
  new variables
\end{enumerate}

\begin{enumerate}
\def\labelenumi{\arabic{enumi}.}
\setcounter{enumi}{2}
\tightlist
\item
  \textbf{BUILD MODELS (25 Points)} Using the training data set, build
  at least three different multiple linear regression models, using
  different variables (or the same variables with different
  transformations). Since we have not yet covered automated variable
  selection methods, you should select the variables manually (unless
  you previously learned Forward or Stepwise selection, etc.). Since you
  manually selected a variable for inclusion into the model or exclusion
  into the model, indicate why this was done. Discuss the coefficients
  in the models, do they make sense? For example, if a team hits a lot
  of Home Runs, it would be reasonably expected that such a team would
  win more games. However, if the coefficient is negative (suggesting
  that the team would lose more games), then that needs to be discussed.
  Are you keeping the model even though it is counter intuitive? Why?
  The boss needs to know.
\item
  \textbf{SELECT MODELS (25 Points)} Decide on the criteria for
  selecting the best multiple linear regression model. Will you select a
  model with slightly worse performance if it makes more sense or is
  more parsimonious? Discuss why you selected your model. For the
  multiple linear regression model, will you use a metric such as
  Adjusted R2 , RMSE, etc.? Be sure to explain how you can make
  inferences from the model, discuss multi-collinearity issues (if any),
  and discuss other relevant model output. Using the training data set,
  evaluate the multiple linear regression model based on (a) mean
  squared error, (b) R2, (c) F-statistic, and (d) residual plots. Make
  predictions using the evaluation data set.
\end{enumerate}

\hypertarget{evaluation}{%
\subsubsection{Evaluation}\label{evaluation}}

\textbf{Load the data}

\begin{Shaded}
\begin{Highlighting}[]
\NormalTok{df\_train }\OtherTok{\textless{}{-}} \FunctionTok{read.csv}\NormalTok{(}\StringTok{"https://raw.githubusercontent.com/melbow2424/Data621\_HW1/main/moneyball{-}training{-}data.csv"}\NormalTok{)}

\NormalTok{df\_evaluation }\OtherTok{\textless{}{-}} \FunctionTok{read.csv}\NormalTok{(}\StringTok{"https://raw.githubusercontent.com/melbow2424/Data621\_HW1/main/moneyball{-}evaluation{-}data.csv"}\NormalTok{)}
\end{Highlighting}
\end{Shaded}

\textbf{Review Data}

\begin{Shaded}
\begin{Highlighting}[]
\FunctionTok{skim}\NormalTok{(df\_train)}
\end{Highlighting}
\end{Shaded}

\begin{longtable}[]{@{}ll@{}}
\caption{Data summary}\tabularnewline
\toprule()
\endhead
Name & df\_train \\
Number of rows & 2276 \\
Number of columns & 17 \\
\_\_\_\_\_\_\_\_\_\_\_\_\_\_\_\_\_\_\_\_\_\_\_ & \\
Column type frequency: & \\
numeric & 17 \\
\_\_\_\_\_\_\_\_\_\_\_\_\_\_\_\_\_\_\_\_\_\_\_\_ & \\
Group variables & None \\
\bottomrule()
\end{longtable}

\textbf{Variable type: numeric}

\begin{longtable}[]{@{}
  >{\raggedright\arraybackslash}p{(\columnwidth - 20\tabcolsep) * \real{0.1753}}
  >{\raggedleft\arraybackslash}p{(\columnwidth - 20\tabcolsep) * \real{0.1031}}
  >{\raggedleft\arraybackslash}p{(\columnwidth - 20\tabcolsep) * \real{0.1443}}
  >{\raggedleft\arraybackslash}p{(\columnwidth - 20\tabcolsep) * \real{0.0825}}
  >{\raggedleft\arraybackslash}p{(\columnwidth - 20\tabcolsep) * \real{0.0825}}
  >{\raggedleft\arraybackslash}p{(\columnwidth - 20\tabcolsep) * \real{0.0515}}
  >{\raggedleft\arraybackslash}p{(\columnwidth - 20\tabcolsep) * \real{0.0825}}
  >{\raggedleft\arraybackslash}p{(\columnwidth - 20\tabcolsep) * \real{0.0722}}
  >{\raggedleft\arraybackslash}p{(\columnwidth - 20\tabcolsep) * \real{0.0825}}
  >{\raggedleft\arraybackslash}p{(\columnwidth - 20\tabcolsep) * \real{0.0619}}
  >{\raggedright\arraybackslash}p{(\columnwidth - 20\tabcolsep) * \real{0.0619}}@{}}
\toprule()
\begin{minipage}[b]{\linewidth}\raggedright
skim\_variable
\end{minipage} & \begin{minipage}[b]{\linewidth}\raggedleft
n\_missing
\end{minipage} & \begin{minipage}[b]{\linewidth}\raggedleft
complete\_rate
\end{minipage} & \begin{minipage}[b]{\linewidth}\raggedleft
mean
\end{minipage} & \begin{minipage}[b]{\linewidth}\raggedleft
sd
\end{minipage} & \begin{minipage}[b]{\linewidth}\raggedleft
p0
\end{minipage} & \begin{minipage}[b]{\linewidth}\raggedleft
p25
\end{minipage} & \begin{minipage}[b]{\linewidth}\raggedleft
p50
\end{minipage} & \begin{minipage}[b]{\linewidth}\raggedleft
p75
\end{minipage} & \begin{minipage}[b]{\linewidth}\raggedleft
p100
\end{minipage} & \begin{minipage}[b]{\linewidth}\raggedright
hist
\end{minipage} \\
\midrule()
\endhead
INDEX & 0 & 1.00 & 1268.46 & 736.35 & 1 & 630.75 & 1270.5 & 1915.50 &
2535 & ▇▇▇▇▇ \\
TARGET\_WINS & 0 & 1.00 & 80.79 & 15.75 & 0 & 71.00 & 82.0 & 92.00 & 146
& ▁▁▇▅▁ \\
TEAM\_BATTING\_H & 0 & 1.00 & 1469.27 & 144.59 & 891 & 1383.00 & 1454.0
& 1537.25 & 2554 & ▁▇▂▁▁ \\
TEAM\_BATTING\_2B & 0 & 1.00 & 241.25 & 46.80 & 69 & 208.00 & 238.0 &
273.00 & 458 & ▁▆▇▂▁ \\
TEAM\_BATTING\_3B & 0 & 1.00 & 55.25 & 27.94 & 0 & 34.00 & 47.0 & 72.00
& 223 & ▇▇▂▁▁ \\
TEAM\_BATTING\_HR & 0 & 1.00 & 99.61 & 60.55 & 0 & 42.00 & 102.0 &
147.00 & 264 & ▇▆▇▅▁ \\
TEAM\_BATTING\_BB & 0 & 1.00 & 501.56 & 122.67 & 0 & 451.00 & 512.0 &
580.00 & 878 & ▁▁▇▇▁ \\
TEAM\_BATTING\_SO & 102 & 0.96 & 735.61 & 248.53 & 0 & 548.00 & 750.0 &
930.00 & 1399 & ▁▆▇▇▁ \\
TEAM\_BASERUN\_SB & 131 & 0.94 & 124.76 & 87.79 & 0 & 66.00 & 101.0 &
156.00 & 697 & ▇▃▁▁▁ \\
TEAM\_BASERUN\_CS & 772 & 0.66 & 52.80 & 22.96 & 0 & 38.00 & 49.0 &
62.00 & 201 & ▃▇▁▁▁ \\
TEAM\_BATTING\_HBP & 2085 & 0.08 & 59.36 & 12.97 & 29 & 50.50 & 58.0 &
67.00 & 95 & ▂▇▇▅▁ \\
TEAM\_PITCHING\_H & 0 & 1.00 & 1779.21 & 1406.84 & 1137 & 1419.00 &
1518.0 & 1682.50 & 30132 & ▇▁▁▁▁ \\
TEAM\_PITCHING\_HR & 0 & 1.00 & 105.70 & 61.30 & 0 & 50.00 & 107.0 &
150.00 & 343 & ▇▇▆▁▁ \\
TEAM\_PITCHING\_BB & 0 & 1.00 & 553.01 & 166.36 & 0 & 476.00 & 536.5 &
611.00 & 3645 & ▇▁▁▁▁ \\
TEAM\_PITCHING\_SO & 102 & 0.96 & 817.73 & 553.09 & 0 & 615.00 & 813.5 &
968.00 & 19278 & ▇▁▁▁▁ \\
TEAM\_FIELDING\_E & 0 & 1.00 & 246.48 & 227.77 & 65 & 127.00 & 159.0 &
249.25 & 1898 & ▇▁▁▁▁ \\
TEAM\_FIELDING\_DP & 286 & 0.87 & 146.39 & 26.23 & 52 & 131.00 & 149.0 &
164.00 & 228 & ▁▂▇▆▁ \\
\bottomrule()
\end{longtable}

\textbf{Get the Means of columns in Training Data}

\begin{Shaded}
\begin{Highlighting}[]
\NormalTok{train\_means}\OtherTok{\textless{}{-}}\FunctionTok{sapply}\NormalTok{(df\_train, }\ControlFlowTok{function}\NormalTok{(x) }\FunctionTok{round}\NormalTok{(}\FunctionTok{mean}\NormalTok{(x, }\AttributeTok{na.rm =} \ConstantTok{TRUE}\NormalTok{)))}
\NormalTok{train\_means}
\end{Highlighting}
\end{Shaded}

\begin{verbatim}
##            INDEX      TARGET_WINS   TEAM_BATTING_H  TEAM_BATTING_2B 
##             1268               81             1469              241 
##  TEAM_BATTING_3B  TEAM_BATTING_HR  TEAM_BATTING_BB  TEAM_BATTING_SO 
##               55              100              502              736 
##  TEAM_BASERUN_SB  TEAM_BASERUN_CS TEAM_BATTING_HBP  TEAM_PITCHING_H 
##              125               53               59             1779 
## TEAM_PITCHING_HR TEAM_PITCHING_BB TEAM_PITCHING_SO  TEAM_FIELDING_E 
##              106              553              818              246 
## TEAM_FIELDING_DP 
##              146
\end{verbatim}

\textbf{Get the Medians of columns in training data}

\begin{Shaded}
\begin{Highlighting}[]
\NormalTok{train\_medians}\OtherTok{\textless{}{-}}\FunctionTok{sapply}\NormalTok{(df\_train, }\ControlFlowTok{function}\NormalTok{(x) }\FunctionTok{round}\NormalTok{(}\FunctionTok{median}\NormalTok{(x, }\AttributeTok{na.rm =} \ConstantTok{TRUE}\NormalTok{)))}
\NormalTok{train\_medians}
\end{Highlighting}
\end{Shaded}

\begin{verbatim}
##            INDEX      TARGET_WINS   TEAM_BATTING_H  TEAM_BATTING_2B 
##             1270               82             1454              238 
##  TEAM_BATTING_3B  TEAM_BATTING_HR  TEAM_BATTING_BB  TEAM_BATTING_SO 
##               47              102              512              750 
##  TEAM_BASERUN_SB  TEAM_BASERUN_CS TEAM_BATTING_HBP  TEAM_PITCHING_H 
##              101               49               58             1518 
## TEAM_PITCHING_HR TEAM_PITCHING_BB TEAM_PITCHING_SO  TEAM_FIELDING_E 
##              107              536              814              159 
## TEAM_FIELDING_DP 
##              149
\end{verbatim}

\textbf{Replace NA values in columns with their respective Mean}

\begin{Shaded}
\begin{Highlighting}[]
\CommentTok{\# Replace NA values in \textquotesingle{}column\_name\textquotesingle{} with \textquotesingle{}mean\textquotesingle{}}
\NormalTok{df\_train }\OtherTok{\textless{}{-}}\NormalTok{ df\_train }\SpecialCharTok{\%\textgreater{}\%}
  \FunctionTok{mutate}\NormalTok{(}\AttributeTok{TEAM\_BATTING\_SO =}
           \FunctionTok{ifelse}\NormalTok{(}\FunctionTok{is.na}\NormalTok{(TEAM\_BATTING\_SO),}
\NormalTok{                  train\_means[}\DecValTok{8}\NormalTok{],TEAM\_BATTING\_SO))}\SpecialCharTok{\%\textgreater{}\%} 
  \FunctionTok{mutate}\NormalTok{(}\AttributeTok{TEAM\_BASERUN\_SB =} 
           \FunctionTok{ifelse}\NormalTok{(}\FunctionTok{is.na}\NormalTok{(TEAM\_BASERUN\_SB),}
\NormalTok{                  train\_means[}\DecValTok{9}\NormalTok{], TEAM\_BASERUN\_SB))}\SpecialCharTok{\%\textgreater{}\%}
  \FunctionTok{mutate}\NormalTok{(}\AttributeTok{TEAM\_BASERUN\_CS =}
           \FunctionTok{ifelse}\NormalTok{(}\FunctionTok{is.na}\NormalTok{(TEAM\_BASERUN\_CS),}
\NormalTok{                  train\_means[}\DecValTok{10}\NormalTok{], TEAM\_BASERUN\_CS))}\SpecialCharTok{\%\textgreater{}\%}
  \FunctionTok{mutate}\NormalTok{(}\AttributeTok{TEAM\_BATTING\_HBP =} 
           \FunctionTok{ifelse}\NormalTok{(}\FunctionTok{is.na}\NormalTok{(TEAM\_BATTING\_HBP),}
\NormalTok{                  train\_means[}\DecValTok{11}\NormalTok{],TEAM\_BATTING\_HBP))}\SpecialCharTok{\%\textgreater{}\%}
  \FunctionTok{mutate}\NormalTok{(}\AttributeTok{TEAM\_PITCHING\_SO =}
           \FunctionTok{ifelse}\NormalTok{(}\FunctionTok{is.na}\NormalTok{(TEAM\_PITCHING\_SO),}
\NormalTok{                  train\_means[}\DecValTok{15}\NormalTok{], TEAM\_PITCHING\_SO))}\SpecialCharTok{\%\textgreater{}\%}
  \FunctionTok{mutate}\NormalTok{(}\AttributeTok{TEAM\_FIELDING\_DP =}
           \FunctionTok{ifelse}\NormalTok{(}\FunctionTok{is.na}\NormalTok{(TEAM\_FIELDING\_DP),}
\NormalTok{                  train\_means[}\DecValTok{17}\NormalTok{], TEAM\_FIELDING\_DP))}
\end{Highlighting}
\end{Shaded}

\textbf{Replace NA values with their respective Medians}

\begin{Shaded}
\begin{Highlighting}[]
\CommentTok{\# Replace NA values in \textquotesingle{}column\_name\textquotesingle{} with \textquotesingle{}median\textquotesingle{}}
\CommentTok{\# df\_train \textless{}{-} df\_train \%\textgreater{}\%}
\CommentTok{\#   mutate(TEAM\_BATTING\_SO =}
\CommentTok{\#            ifelse(is.na(TEAM\_BATTING\_SO),}
\CommentTok{\#                   train\_medians[8],TEAM\_BATTING\_SO))\%\textgreater{}\% }
\CommentTok{\#   mutate(TEAM\_BASERUN\_SB = }
\CommentTok{\#            ifelse(is.na(TEAM\_BASERUN\_SB),}
\CommentTok{\#                   train\_medians[9], TEAM\_BASERUN\_SB))\%\textgreater{}\%}
\CommentTok{\#   mutate(TEAM\_BASERUN\_CS =}
\CommentTok{\#            ifelse(is.na(TEAM\_BASERUN\_CS),}
\CommentTok{\#                   train\_medians[10], TEAM\_BASERUN\_CS))\%\textgreater{}\%}
\CommentTok{\#   mutate(TEAM\_BATTING\_HBP = }
\CommentTok{\#            ifelse(is.na(TEAM\_BATTING\_HBP),}
\CommentTok{\#                   train\_medians[11],TEAM\_BATTING\_HBP))\%\textgreater{}\%}
\CommentTok{\#   mutate(TEAM\_PITCHING\_SO =}
\CommentTok{\#            ifelse(is.na(TEAM\_PITCHING\_SO),}
\CommentTok{\#                   train\_medians[15], TEAM\_PITCHING\_SO))\%\textgreater{}\%}
\CommentTok{\#   mutate(TEAM\_FIELDING\_DP =}
\CommentTok{\#            ifelse(is.na(TEAM\_FIELDING\_DP),}
\CommentTok{\#                   train\_medians[17], TEAM\_FIELDING\_DP))}
\end{Highlighting}
\end{Shaded}

\textbf{Note:}

\begin{verbatim}
      While deciding on whether to use 'Mean' or 'Median' both 
      codes were generated. Unused replacement is left 
      commented out, since only one can be applied at a time.
\end{verbatim}

\begin{Shaded}
\begin{Highlighting}[]
\FunctionTok{skim}\NormalTok{(df\_train)}
\end{Highlighting}
\end{Shaded}

\begin{longtable}[]{@{}ll@{}}
\caption{Data summary}\tabularnewline
\toprule()
\endhead
Name & df\_train \\
Number of rows & 2276 \\
Number of columns & 17 \\
\_\_\_\_\_\_\_\_\_\_\_\_\_\_\_\_\_\_\_\_\_\_\_ & \\
Column type frequency: & \\
numeric & 17 \\
\_\_\_\_\_\_\_\_\_\_\_\_\_\_\_\_\_\_\_\_\_\_\_\_ & \\
Group variables & None \\
\bottomrule()
\end{longtable}

\textbf{Variable type: numeric}

\begin{longtable}[]{@{}
  >{\raggedright\arraybackslash}p{(\columnwidth - 20\tabcolsep) * \real{0.1753}}
  >{\raggedleft\arraybackslash}p{(\columnwidth - 20\tabcolsep) * \real{0.1031}}
  >{\raggedleft\arraybackslash}p{(\columnwidth - 20\tabcolsep) * \real{0.1443}}
  >{\raggedleft\arraybackslash}p{(\columnwidth - 20\tabcolsep) * \real{0.0825}}
  >{\raggedleft\arraybackslash}p{(\columnwidth - 20\tabcolsep) * \real{0.0825}}
  >{\raggedleft\arraybackslash}p{(\columnwidth - 20\tabcolsep) * \real{0.0515}}
  >{\raggedleft\arraybackslash}p{(\columnwidth - 20\tabcolsep) * \real{0.0825}}
  >{\raggedleft\arraybackslash}p{(\columnwidth - 20\tabcolsep) * \real{0.0722}}
  >{\raggedleft\arraybackslash}p{(\columnwidth - 20\tabcolsep) * \real{0.0825}}
  >{\raggedleft\arraybackslash}p{(\columnwidth - 20\tabcolsep) * \real{0.0619}}
  >{\raggedright\arraybackslash}p{(\columnwidth - 20\tabcolsep) * \real{0.0619}}@{}}
\toprule()
\begin{minipage}[b]{\linewidth}\raggedright
skim\_variable
\end{minipage} & \begin{minipage}[b]{\linewidth}\raggedleft
n\_missing
\end{minipage} & \begin{minipage}[b]{\linewidth}\raggedleft
complete\_rate
\end{minipage} & \begin{minipage}[b]{\linewidth}\raggedleft
mean
\end{minipage} & \begin{minipage}[b]{\linewidth}\raggedleft
sd
\end{minipage} & \begin{minipage}[b]{\linewidth}\raggedleft
p0
\end{minipage} & \begin{minipage}[b]{\linewidth}\raggedleft
p25
\end{minipage} & \begin{minipage}[b]{\linewidth}\raggedleft
p50
\end{minipage} & \begin{minipage}[b]{\linewidth}\raggedleft
p75
\end{minipage} & \begin{minipage}[b]{\linewidth}\raggedleft
p100
\end{minipage} & \begin{minipage}[b]{\linewidth}\raggedright
hist
\end{minipage} \\
\midrule()
\endhead
INDEX & 0 & 1 & 1268.46 & 736.35 & 1 & 630.75 & 1270.5 & 1915.50 & 2535
& ▇▇▇▇▇ \\
TARGET\_WINS & 0 & 1 & 80.79 & 15.75 & 0 & 71.00 & 82.0 & 92.00 & 146 &
▁▁▇▅▁ \\
TEAM\_BATTING\_H & 0 & 1 & 1469.27 & 144.59 & 891 & 1383.00 & 1454.0 &
1537.25 & 2554 & ▁▇▂▁▁ \\
TEAM\_BATTING\_2B & 0 & 1 & 241.25 & 46.80 & 69 & 208.00 & 238.0 &
273.00 & 458 & ▁▆▇▂▁ \\
TEAM\_BATTING\_3B & 0 & 1 & 55.25 & 27.94 & 0 & 34.00 & 47.0 & 72.00 &
223 & ▇▇▂▁▁ \\
TEAM\_BATTING\_HR & 0 & 1 & 99.61 & 60.55 & 0 & 42.00 & 102.0 & 147.00 &
264 & ▇▆▇▅▁ \\
TEAM\_BATTING\_BB & 0 & 1 & 501.56 & 122.67 & 0 & 451.00 & 512.0 &
580.00 & 878 & ▁▁▇▇▁ \\
TEAM\_BATTING\_SO & 0 & 1 & 735.62 & 242.89 & 0 & 556.75 & 736.0 &
925.00 & 1399 & ▁▅▇▇▁ \\
TEAM\_BASERUN\_SB & 0 & 1 & 124.78 & 85.23 & 0 & 67.00 & 106.0 & 151.00
& 697 & ▇▂▁▁▁ \\
TEAM\_BASERUN\_CS & 0 & 1 & 52.87 & 18.66 & 0 & 44.00 & 53.0 & 54.25 &
201 & ▂▇▁▁▁ \\
TEAM\_BATTING\_HBP & 0 & 1 & 59.03 & 3.75 & 29 & 59.00 & 59.0 & 59.00 &
95 & ▁▁▇▁▁ \\
TEAM\_PITCHING\_H & 0 & 1 & 1779.21 & 1406.84 & 1137 & 1419.00 & 1518.0
& 1682.50 & 30132 & ▇▁▁▁▁ \\
TEAM\_PITCHING\_HR & 0 & 1 & 105.70 & 61.30 & 0 & 50.00 & 107.0 & 150.00
& 343 & ▇▇▆▁▁ \\
TEAM\_PITCHING\_BB & 0 & 1 & 553.01 & 166.36 & 0 & 476.00 & 536.5 &
611.00 & 3645 & ▇▁▁▁▁ \\
TEAM\_PITCHING\_SO & 0 & 1 & 817.74 & 540.54 & 0 & 626.00 & 818.0 &
957.00 & 19278 & ▇▁▁▁▁ \\
TEAM\_FIELDING\_E & 0 & 1 & 246.48 & 227.77 & 65 & 127.00 & 159.0 &
249.25 & 1898 & ▇▁▁▁▁ \\
TEAM\_FIELDING\_DP & 0 & 1 & 146.34 & 24.52 & 52 & 134.00 & 146.0 &
161.25 & 228 & ▁▂▇▅▁ \\
\bottomrule()
\end{longtable}

\hypertarget{reference}{%
\subsubsection{Reference}\label{reference}}

\begin{itemize}
\tightlist
\item
  ``Pythagorean Theorem of Baseball.'' Baseball Reference,
  \url{https://www.baseball-reference.com/bullpen/Pythagorean_Theorem_of_Baseball}.
  Accessed 11 September 2023.
\item
  No author listed. ``Pythagorean Expectation in Major League
  Baseball.'' Digital Commons @ Cal Poly,
  \url{https://digitalcommons.calpoly.edu/cgi/viewcontent.cgi?article=1067\&context=statsp}.
  Accessed 11 September 2023.
\end{itemize}

\end{document}
